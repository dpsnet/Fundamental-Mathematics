% 维度数学理论论文模板
% Mathematical Theory of Dimension - Paper Template

\documentclass[12pt,a4paper]{amsart}

% 基础包
\usepackage[utf8]{inputenc}
\usepackage[T1]{fontenc}
\usepackage{amsmath,amssymb,amsthm}
\usepackage{geometry}
\usepackage{hyperref}
\usepackage{graphicx}
\usepackage{booktabs}
\usepackage{enumitem}

% 页面设置
\geometry{margin=2.5cm}

% 定理环境
\theoremstyle{plain}
\newtheorem{theorem}{Theorem}[section]
\newtheorem{lemma}[theorem]{Lemma}
\newtheorem{proposition}[theorem]{Proposition}
\newtheorem{corollary}[theorem]{Corollary}
\newtheorem{conjecture}[theorem]{Conjecture}

\theoremstyle{definition}
\newtheorem{definition}[theorem]{Definition}
\newtheorem{example}[theorem]{Example}
\newtheorem{remark}[theorem]{Remark}
\newtheorem{assumption}[theorem]{Assumption}

% 引用设置
\hypersetup{
    colorlinks=true,
    linkcolor=blue,
    citecolor=red,
    urlcolor=blue
}

% 标题信息(由具体论文填充)
\title[Short Title]{Full Title of the Paper}
\author[Author Name]{Author Name}
\address{Institution Address}
\email{email@institution.edu}
\date{\today}

% 自定义命令
\newcommand{\R}{\mathbb{R}}
\newcommand{\N}{\mathbb{N}}
\newcommand{\Z}{\mathbb{Z}}
\newcommand{\C}{\mathbb{C}}
\newcommand{\HH}{\mathcal{H}}
\newcommand{\eps}{\varepsilon}
\DeclareMathOperator{\dimH}{dim_H}
\DeclareMathOperator{\dimB}{dim_B}
\DeclareMathOperator{\diam}{diam}
\DeclareMathOperator{\supp}{supp}
\DeclareMathOperator{\tr}{tr}

\begin{document}

\begin{abstract}
% 摘要(150-250词)
This paper establishes rigorous mathematical results on [topic]. 
We prove [main theorem] and provide numerical validation. 
Our work corrects previous claims in the M-0 series and provides 
a solid foundation for [application].

\textbf{Keywords:} keyword1, keyword2, keyword3, keyword4
\end{abstract}

\maketitle

\tableofcontents

%==============================================================================
\section{Introduction}\label{sec:intro}
%==============================================================================

\subsection{Background}
[背景介绍]

\subsection{Main Results}
[主要结果陈述]

\begin{theorem}[Main Theorem]\label{thm:main}
Statement of the main theorem.
\end{theorem}

\subsection{Relation to Previous Work}
[与先前工作的关系,特别是 M-0 系列的纠正]

%==============================================================================
\section{Preliminaries}\label{sec:prelim}
%==============================================================================

\subsection{Notation}
[符号定义]

\subsection{Key Definitions}
[关键定义]

\begin{definition}[Key Concept]\label{def:key}
Definition of the key concept.
\end{definition}

%==============================================================================
\section{Main Theory}\label{sec:main}
%==============================================================================

[主要理论内容]

\begin{theorem}[Important Result]\label{thm:important}
Statement of an important result.
\end{theorem}

\begin{proof}
Proof of the theorem.
\end{proof}

\begin{corollary}[Corollary]\label{cor:corollary}
Statement of corollary.
\end{corollary}

%==============================================================================
\section{Numerical Validation}\label{sec:numerical}
%==============================================================================

[数值验证部分,如适用]

\begin{table}[htbp]
\centering
\caption{Numerical Results}\label{tab:numerical}
\begin{tabular}{ccc}
\toprule
Parameter & Value & Error \\
\midrule
$\alpha$ & 0.5 & $<0.01$ \\
$A$ & 1.5 & $<0.1$ \\
$d^*$ & 0.617 & $2.8\%$ \\
\bottomrule
\end{tabular}
\end{table}

%==============================================================================
\section{Applications}\label{sec:applications}
%==============================================================================

[应用部分]

%==============================================================================
\section{Conclusion}\label{sec:conclusion}
%==============================================================================

[总结]

\subsection*{Acknowledgments}
We acknowledge the inspiration from the M-0 series while correcting 
its mathematical inaccuracies. The rigorous foundation of this work 
is based on classical mathematical literature.

%==============================================================================
\bibliographystyle{amsplain}
\begin{thebibliography}{99}

\bibitem{JonssonWallin1984}
A. Jonsson and H. Wallin,
\textit{Function spaces on subsets of $\mathbb{R}^n$},
Math. Rep. 2 (1984), no. 1, xiv+221.

\bibitem{Borwein2002}
P. Borwein,
\textit{The Prouhet-Tarry-Escott problem},
Springer, 2002.

\bibitem{HindrySilverman2000}
M. Hindry and J. Silverman,
\textit{Diophantine geometry},
Springer-Verlag, New York, 2000.

\bibitem{Lapidus2013}
M. L. Lapidus and M. van Frankenhuijsen,
\textit{Fractal geometry, complex dimensions and zeta functions},
Second edition, Springer, New York, 2013.

\bibitem{Triebel1997}
H. Triebel,
\textit{Fractals and spectra},
Birkh\"auser Verlag, Basel, 1997.

\bibitem{Valiant1979}
L. G. Valiant,
The complexity of computing the permanent,
\textit{Theoret. Comput. Sci.} 8 (1979), no. 2, 189--201.

\bibitem{MulmuleySohoni2001}
K. Mulmuley and M. Sohoni,
Geometric complexity theory. I. An approach to the P vs. NP and related problems,
\textit{SIAM J. Comput.} 31 (2001), no. 2, 496--526.

\bibitem{Kigami2001}
J. Kigami,
\textit{Analysis on fractals},
Cambridge University Press, Cambridge, 2001.

\bibitem{Wilson1974}
K. G. Wilson,
Confinement of quarks,
\textit{Phys. Rev. D} 10 (1974), 2445--2459.

\bibitem{Schmidt1972}
W. M. Schmidt,
Norm form equations,
\textit{Ann. of Math.} (2) 96 (1972), 526--551.

\bibitem{DiamondShurman2005}
F. Diamond and J. Shurman,
\textit{A first course in modular forms},
Springer, New York, 2005.

\bibitem{GawedzkiKupiainen1986}
K. Gawedzki and A. Kupiainen,
Asymptotic freedom beyond perturbation theory,
In: Critical phenomena, random systems, gauge theories (Les Houches, 1984), 
North-Holland, Amsterdam, 1986, 185--292.

\end{thebibliography}

\end{document}
